El algoritmo constructivo goloso que diseñamos está basado en el algoritmo de \emph{Dijkstra} para encontrar caminos mínimos en un grafo. 

Supongamos que queremos resolver CACM para un grafo $G$ $=$ ($V$,$E$) donde $E$ = $(e_1, e_2, ..., e_m)$. Lo que nuestro algoritmo hace es, para un conjunto de valores $\{p$ $\in \mathbb{R}$/ $0 \leq p \leq 1\}$, correr el algoritmo de \emph{Dijkstra} para el grafo $G'$ $=$ ($V$, $E'$) donde $E'$ tiene las mismas aristas que $E$ excepto que tiene un único peso, el cual está determinado por la siguiente función $f: (\mathbb{N} x \mathbb{N}) \Rightarrow \mathbb{N}$

\begin{center}
$f(e_i,p) = (1-p)*w_1(e_i) + p*w_2(e_i)$
\end{center}

Esta función pondera los pesos del grafo $G$ asignándole más importancia al peso $w_1$ para los valores más bajos de $p$ o a $w_2$ para los valores más altos. La noción intuitiva es que al dar m\'as importancia a uno de los pesos el algoritmo encontrara un camino menor en ese peso.

Lo que nuestro algoritmo hace es, para $p$ entre 0 y 1 encontrar el $p$ más grande tal que al correr el algoritmo de \emph{Dijkstra} en $G'$ desde el nodo $u$, el peso $w_1$ siga estando acotado por la constante $K$ en el camino de $u$ a $v$. Hacemos esto mediante una búsqueda binaria en los valores de $p$.

Esto tiene sentido ya que nuestro algoritmo tiene dos propiedades que demostraremos luego:

$\mathbf{1}$ El algoritmo siempre encuentra una solución factible si existe.

$\mathbf{2}$ El peso $w_1$ del camino encontrado en función de $p$ es creciente, y el peso $w_2$ decreciente.

Primero con $\mathbf{1}$ nos garantizamos de siempre devolver una solución factible.
Gracias a $\mathbf{2}$ podemos ver que sea $p'$, el $p$ más grande tal que el peso $w_1$ siga estando acotado por la constante $K$, su camino asociado tendrá el menor (o igual) peso en $w_2$ entre todos los caminos factibles que encuentra el algoritmo para algún $p$.

Para demostrar esto vemos dos cosas:

1 - Los $p'' > p'$ no son factible ya que $p'$ era el m\'aximo $p$  tal que el peso $w_1$ siga estando acotado por la constante $K$. Si $p''$ es factible, o sea el peso $w_1$ esta acotado por la constante $K$, $p'$ no ser\'ia el m\'aximo, absurdo.

2- Los $p'' < p'$ tendrán peso $w_1$ mayor o igual al de $p'$ ya que por $\mathbf{2}$ $w_2$ es decreciente en $p$.


Demostraci\'on $\mathbf{1}:$

Solo analizaremos el caso en que existe un camino factible, ya que si no existiese nuestro algoritmo nunca podr\'ia dar uno.
Nuestro algoritmo busca el m\'aximo $p \in [0,1]$ tal que el camino asociado a \'el sea factible. Con mostrar que siempre habrá un $p$ tal que su camino asociado es factible, existe el m\'aximo.

Tomando $p=0$, $f(e_i,0) = (1-0)*w_1(e_i) + 0*w_2(e_i) = w_1(e_i)$

O sea ser\'ia buscar el camino m\'inimo en $w_1$. Si este camino no es factible (peso $w_1 < k$), no existirá ningún camino factible (ya que todos tendrán peso mayor o igual al m\'inimo) absurdo porque estamos en el caso que existe un camino factible.

Demostraci\'on $\mathbf{2}:$

Primero veamos que el peso $w_2$ del camino encontrado en función de $p$ es decreciente.

Tomemos $p_1 < p_2$ cuyos caminos asociados son $C_1$ y $C_2$ correspondientemente.

Queremos ver que $w_2(C_1) > w_2(C_2)$.

Primero veamos que $\frac{p_1}{1-p_1} < \frac{p_2}{1-p_2}$

$ p_1* (1-p_2) < p_2 * (1-p_2) < p_2 * (1-p_1)$

$ p_1 - (1-p_2) < p_2 * (1-p_1)$

$ \frac{p_1}{1-p_1} < \frac{p_2}{1-p_2} $ suponiendo que ninguno es o 0 o 1.


Sabemos que $C_1$ es el camino óptimo con $p=p_1$, particularmente es menor o igual que $C_2$:

$(1-p_1)*w_1(C_1) + p_1*w_2(C_1) \leq (1-p_1)*w_1(C_2) + p_1*w_2(C_2) $

$(1-p_1)*w_1(C_1) - (1-p_1)*w_1(C_2) \leq  p_1*w_2(C_2) - p_1*w_2(C_1) $

$(1-p_1)*(w_1(C_1) - w_1(C_2)) \leq  p_1*(w_2(C_2) - w_2(C_1)) $

$\frac{(1-p_1)}{p_1}*(w_1(C_1) - w_1(C_2)) \leq  w_2(C_2) - w_2(C_1) $

De la misma forma que en el caso anterior, sabemos que $C_2$ es el camino óptimo con $p=p_2$, particularmente es menor o igual que $C_1$:

$(1-p_2)*w_1(C_1) + p_2*w_2(C_1) \geq (1-p_2)*w_1(C_2) + p_2*w_2(C_2) $

$(1-p_2)*w_1(C_1) - (1-p_2)*w_1(C_2) \geq  p_2*w_2(C_2) - p_2*w_2(C_1) $

$(1-p_2)*(w_1(C_1) - w_1(C_2)) \geq  p_2*(w_2(C_2) - w_2(C_1)) $

$\frac{(1-p_2)}{p_2}*(w_1(C_1) - w_1(C_2)) \geq  w_2(C_2) - w_2(C_1) $

Por los que nos queda:

$\frac{(1-p_1)}{p_1}*(w_1(C_1) - w_1(C_2)) \leq  w_2(C_2) - w_2(C_1) \leq \frac{(1-p_2)}{p_2}*(w_1(C_1) - w_1(C_2))$

$\frac{(1-p_1)}{p_1}*(w_1(C_1) - w_1(C_2)) \leq \frac{(1-p_2)}{p_2}*(w_1(C_1) - w_1(C_2))$

$ 0 \leq (\frac{(1-p_2)}{p_2} - \frac{(1-p_1)}{p_1})*(w_1(C_1) - w_1(C_2))$

Como vimos al principio $\frac{p_1}{1-p_1} < \frac{p_2}{1-p_2}$, por lo que $0 < \frac{(1-p_2)}{p_2} - \frac{(1-p_1)}{p_1} $

Entonces $ 0 \leq (w_1(C_1) - w_1(C_2))$

$w_2(C_1) > w_2(C_2)$

Demostrar que $w_1$ es creciente es análogo a lo demostrado anteriormente. Una forma de verlo es tomar $t$ = $1-p$ por lo que se ``invierten'' $w_1$ y $w_2$. Usando exactamente el mismo razonamiento que la última demostración llegamos a que $w_1$ es decreciente en $t$, y como $t$ = $1-p$ significa que es creciente en $p$.

Vale aclarar que la componente golosa de este algoritmo se hace presente cuando corremos el algoritmo de \emph{Dijkstra}, ya que dicho algoritmo es un algoritmo goloso.

A continuación incluimos un pseudocódigo de nuestro algoritmo:

\begin{center}
 \begin{figure}[H]
  \begin{pseudo}
   \Procedure{cacm\_goloso}{grafo $G$, nodo $u$, nodo $v$, entero $K$}
    \State $pInicial \leftarrow 0$, $pFinal \leftarrow 1$, $pMedio$
    \For{$i < Cant\_Iteraciones$}
      \State $pMedio \leftarrow (pInicial + pFinal)/2$
      \State $solucion$ $\leftarrow$ $Dijkstra(G$, $pMedio$, $u$, $v)$
      \If{$w_1(solucion) \leq K$}
	 \State $pInicial \leftarrow pMedio$
      \Else
	 \State $pFinal \leftarrow pMedio$
      \EndIf
    \EndFor
    \If{$w_1(solucion) \leq K$}
      \State $solucion \leftarrow Dijkstra(G, 1)$
    \EndIf
    \State Devolver solución
   \EndProcedure
  \end{pseudo}
 \end{figure}
\end{center}