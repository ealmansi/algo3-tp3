\begin{center}
 \begin{figure}[H]
  \begin{pseudo}
   \Procedure{cacm\_goloso}{grafo $G$, nodo $u$, nodo $v$}
    \State $pInicial \leftarrow 0$, $pFinal \leftarrow 1$, $pMedio$ \Ode{1}
    \For{$i < Cant\_Iteraciones$}
      \State $pMedio \leftarrow (pInicial + pFinal)/2$ \Ode{1}
      \State $solucion$ $\leftarrow$ $Dijkstra(G$, $pMedio$, $u$, $v)$ \Ode{n^2}
      \If{$W_1(solucion) \leq k$} \Ode{1}
	 \State $pInicial \leftarrow pMedio$ \Ode{1}
      \Else
	 \State $pFinal \leftarrow pMedio$ \Ode{1}
      \EndIf
    \EndFor
    \If{$W_1(solucion) \leq k$} \Ode{1}
      \State $solucion \leftarrow Dijkstra(G, 1)$ \Ode{n^2}
    \EndIf
    \State Devolver solucion \Ode{1}
   \EndProcedure
  \end{pseudo}
 \end{figure}
\end{center}

 Como podemos ver el algoritmo esta compuesto por una instruccion $O(1)$ luego un for,y por ultimo 2 $O(1)$ y una vez Dijskstra, que como sabemos tiene complejidad $O(n^2)$. El for itera $Cant\_Iteraciones$ instrucciones $4*O(1) + O(n^2)$. Como $Cant\_Iteraciones$ es un numero acotado, la compleijad del for es $O(n^2)$. Por algebra de ordenes $O(1)+O(n^2)+2*O(1)+O(n^2)=O(n^2)$.  