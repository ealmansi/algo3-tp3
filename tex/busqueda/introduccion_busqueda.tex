En esta sección, implementaremos un algoritmo que utilice la técnica de \emph{búsqueda local}. Para hacer este algoritmo definimos un conjunto de soluciones vecindad para cada solución factible del problema. Luego, un algoritmo de búsqueda local toma una solución factible $S$ del problema a resolver como solución inicial y mejora dicha solución recorriendo en forma iterativa los vecinos de $S$ y reemplazándolo cuando encuentra un vecino mejor. El algoritmo realiza este procedimiento hasta encontrar una solución que no tenga ningún vecino que sea mejor, es decir, hasta encontrar un máximo local.

Básicamente, un algoritmo de búsqueda local debería cumplir con el siguiente pseudocódigo, sea $N(S)$ el conjunto de vecinos de la solución $S$ y $f$ la función objetivo que se quiere minimizar:

\begin{center}
 \begin{figure}[H]
  \begin{pseudo}
   \Procedure{busquedaLocal}{}
   \State Sea $S$ una solución inicial
   \While{$(\exists S' \in N(S))\, f(S')<f(S)$}
      \State $S \leftarrow S'$
   \EndWhile
   \State \textbf{return} $S$
   \EndProcedure
  \end{pseudo}
 \end{figure}
\end{center}
