Un algoritmo GRASP (\emph{Greedy Randomized Adaptative Search Procedure}) es una combinación entre una heurística golosa y un algoritmo de búsqueda local. Los pasos genéricos de GRASP son los siguientes:

\begin{center}
 \begin{figure}[H]
  \begin{pseudo}
  \Procedure{GRASP}{}
    \While{no se alcance el criterio de terminación} 
    \State Obtener $s$ $\in$ $S$ mediante una heurística golosa aleatorizada
    \State Mejorar $s$ mediante búsqueda local
    \State Recordar la mejor solución obtenida hasta el momento
    \EndWhile
    \EndProcedure
  \end{pseudo}
 \end{figure}
\end{center}

Cabe aclarar que podemos generar nuestra solución $s$, la cual obtenemos mediante una heurística golosa aleatorizada, utilizando nuestro algoritmo goloso descripto en la sección \ref{subsub:algoritmos-heuristicos-goloso-desarrollo.tex}, o utilizando Dijkstra sobre la función $\omega_1$, modificándolo como describimos a continuación.

Como nuestro algoritmo goloso original no tiene una componente ``aleatoria'', tuvimos que modificarlo. En GRASP, la componente aleatoria del algoritmo goloso se hace  fabricando en cada paso una Lista Restricta de Candidatos (RCL) y eligiendo de manera aleatoria un candidato de esta lista. Por esta razón, modificamos nuestra implementación del algoritmo de Dijkstra para que lo podamos utilizar. De la misma manera, logramos poder utilizar la nueva implementación de Dijkstra sobre la función de peso $\omega_1$ para tener dos tipos de algoritmos golosos iniciales para el algoritmo de GRASP.

La nueva implementación de Dijkstra que utilizamos es la siguiente:

\begin{center}
 \begin{figure}[H]
  \begin{pseudo}
   \Procedure{Dijkstra}{grafo $G = (V,E)$, $p$, $u$, $v$,$coef\_random$}
    \State $S \leftarrow \emptyset$
    \State $(\forall v \in V)\,dist(v) \leftarrow \infty$
    \State $(\forall v \in V)\,pred(v) \leftarrow NULL$
    \While{$true$}
    \State $minDistU \leftarrow \infty$
    \For{$i =  1$ \textbf{to} $ |V|$}
      \If{$i \in V-S \wedge dist(i) < minDistU$}
	\State $minDistU \leftarrow dist(i)$
	\State $nuevo \leftarrow i$
      \EndIf
    \EndFor
    \If{$minDistU = \infty$}
      \State \textbf{break}
    \EndIf
    \State //Armamos RCL
    \State $candidatos$ $\leftarrow$ $\emptyset$
    \For{$i$ $=$ $1$ \textbf{to} $ |V|$}
      \If{$dist(i)$ $\leq$ $minDistU$ * (1 + $coef\_rand$)}
	\State $candidatos$.push\_back($i$)
      \EndIf
    \EndFor
    \State $u$ $\leftarrow$ $dameUnoAlAzar(candidatos)$
    \State $S \leftarrow S \cup \{nuevo\}$
    \For{$j$ adyacente a $nuevo \wedge j \notin S$}
      \State $distancia \leftarrow (1-p)*w_1(j) + p*w_2(j)$
      \If{$distancia < dist(j)$}
	\State $dist(i) \leftarrow distancia$
      \EndIf
    \EndFor
    \EndWhile
    \State $Solucion \leftarrow \emptyset$
    \State $i \leftarrow v$
    \While{$i \neq 1$}
      \State $Solucion \leftarrow Solucion \cup \{i\}$
      \State $i \leftarrow pred(i)$
    \EndWhile
    \Return Solucion
    \EndProcedure
  \end{pseudo}
 \end{figure}
\end{center}

Nuestro algoritmo GRASP tiene un criterio de terminación híbrido. Este utiliza dos componentes: 

\begin{itemize}
 \item Cantidad máxima de iteraciones
 \item Cantidad de iteraciones sin cambios
\end{itemize}


Utilizar la primer componente nos permite saber que va a terminar en alguna cierta cantidad de iteraciones acotada. De esta manera, nos aseguramos que termina siempre. Y por otro lado, utilizar la segunda componente nos permite asegurarnos de que el algoritmo no va a ejecutarse muchas veces sin conseguir cambios.

Como ahora utilizamos una heurística golosa aleatorizada, un punto importante es la creación de una Lista Restricta de Candidatos (RCL) para poder elegir uno de ellos al azar. Para armar con nuestra RCL, la creamos con los candidatos que cumplan con:

\begin{center}
dist($i$) $\leq$ $minDistU$ * (1 + $coef\_rand$)
\end{center}

Donde $dist(i)$ contiene la distancia del nodo inicial al nodo $i$ medida con la función $f$ y el $p$ pasado por parámetro (ver \ref{subsub:algoritmos-heuristicos-goloso-desarrollo.tex} para entender mejor $dist$), min\_dist\_u es igual al mínimo dist $\forall i$, y coef\_rand es pasado por parámetro en nuestra función de GRASP.

Cabe aclarar que si cambiamos la guarda del $If$ que contiene a la restricción, podemos crear sin mucha complicación otra RCL.

A continuación vamos a presentar el pseudocódigo de nuestro algoritmo GRASP.

\begin{center}
 \begin{figure}[H]
  \begin{pseudo}
  \Procedure{Resolver}{Entrada $e$, entero max\_cant\_it\_sin\_mejora, double coef\_rand, double epsilon}
    \State Salida $s$, $mejor\_s$ $\leftarrow$ golosoResolver($e$, coef\_rand)
    \State $cant\_it\_sin\_mejora$ $\leftarrow$ 0   
    \While{max\_cant\_it\_sin\_mejora $<$ $cant\_it\_sin\_mejora$}
      \State $s$ $\leftarrow$ golosoResolver($e$, coef\_rand)
      \State maximoLocalBusqLocal($e$,$s$)
      \If{\textbf{not} $s$.hay\_solucion}
	\State++$cant\_it\_sin\_mejora$
	\State \textbf{continue}
      \EndIf
      \If{diferencia\_relativa($mejor\_s$, $s$) $<$ epsilon}
	\State ++$cant\_it\_sin\_mejora$
      \Else
	\State $cant\_it\_sin\_mejora$ $\leftarrow$ 0
      \EndIf
      \If{$s.w2$ $<$ mejor\_$s.w2$}
	\State $mejor\_s$ $\leftarrow$ s
      \EndIf
    \EndWhile
    \EndProcedure
  \end{pseudo}
 \end{figure}
\end{center}

\begin{center}
 \begin{figure}[H]
  \begin{pseudo}
  \Procedure{diferencia\_relativa}{salida $mejor\_s$, salida $s$}
    \State \textbf{return}($mejor\_s.w2$ - $s.w2$) / $mejor\_s.w2$
  \EndProcedure
  \end{pseudo}
 \end{figure}
\end{center}