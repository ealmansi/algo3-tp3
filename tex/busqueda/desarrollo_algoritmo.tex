Habiendo explicado las características de nuestra vecindad, a continuación describiremos el algoritmo de busqueda local.

Nuestro algoritmo recibe como entrada el grafo $G$ y una solución inicial $C$. Luego de esto el algoritmo entra en un ciclo en el que se buscan los vecinos de la mejor solución $S$ que se tenga. Si existe un vecino $S'$ mejor que $S$, entonces la mejor solución pasa a ser $S'$ y pasamos a la siguiente iteración. En la iteración en la que no exista un vecino mejor que $S$ entonces se termina la búsqueda.

Cabe aclarar que nuestra solución inicial $C$ la podemos generar utilizando nuestro algoritmo goloso descripto en la sección \ref{subsub:algoritmos-heuristicos-goloso-desarrollo.tex}, o utilizando Dijkstra sobre la función $W_1$.

A continuación incluiremos un pseudocódigo del algoritmo.
\begin{center}
 \begin{figure}[H]
  \begin{pseudo}
   \Procedure{busqueda local}{grafo G, camino sol, c}
   \State $mejorSol \leftarrow sol$\Ode{1}
   \While{$seguirBuscando$}\hfill$c*O(1)$
      \State $buscarVecinoAgregar(mejorSol, mejorVecino, G)$\Ode{n^3}
      \State $buscarVecinoQuitar(mejorSol, mejorVecino, G)$\Ode{n^2}
      \State $buscarVecinoCambiar(mejorSol, mejorVecino, G)$\Ode{n^3}
      \If{$mejorVecino.W_1 > k \vee mejorVecino.W_2 \geq mejorSol.W_2$}\Ode{1}
	\State $seguirBuscando \leftarrow false$\Ode{1}
      \Else
	\State $mejorSol \leftarrow mejorVecino$\Ode{1}
      \EndIf
   \EndWhile
   \State Devolver $mejorSol$\Ode{1}
   \EndProcedure
  \end{pseudo}
 \end{figure}
\end{center}

\begin{flushleft}
 \begin{figure}[H]
  \begin{pseudo}
   \Procedure{buscarVecinoAgregar}{mejorSol, mejorVecino, G}
   \State $i \leftarrow 0$\Ode{1}
   \While{$0 \leq i < cantEjes(mejorSol)-1$}\hfill$n*O(1)$
      \For{$j \in adyacentes(mejorSol_i,G) \cap adyacentes(mejorSol_{i+1},G)$}\hfill$n*O(1)$
	  \State $W_1' \leftarrow mejorSol.W_1 - mejorSol(i,i+1).W_1 + (mejorSol_i,j).W_1 + $
	  \State $(j,mejorSol_{i+1}).W_1$\Ode{1}
	  \State $W_2' \leftarrow mejorSol.W_2 - mejorSol(i,i+1).W_2 + (mejorSol_i,j).W_2 + $
	  \State $(j,mejorSol_{i+1}).W_2$\Ode{1}
	  \If{$W_1' < k \wedge (mejorVecino.W_1 \geq k \vee W_2' < mejorVecino.W_2)$}\Ode{1}
	    \State $mejorVecino = (mejorSol_1, ... ,mejorSol_i, j, mejorSol_{i+1}, ..., mejorSol_{cantEjes})$\Ode{1}
	  \EndIf
      \EndFor
      \State $i \leftarrow i+1$\Ode{1}
   \EndWhile
   \EndProcedure
  \end{pseudo}
 \end{figure}
\end{flushleft}

\begin{flushleft}
 \begin{figure}[H]
  \begin{pseudo}
   \Procedure{buscarVecinoQuitar}{mejorSol, mejorVecino, G}
   \State $i \leftarrow 0$\Ode{1}
   \While{$0 \leq j < cantEjes(mejorSol)-2$}\hfill$n*O(1)$
      \If{$mejorSol_{i+2} \in adyacentes(mejorSol_i,G)$}\Ode{n}
	  \State $W_1' \leftarrow mejorSol.W_1 - mejorSol(i,i+1).W_1 - mejorSol(i+1,i+2).W_1 + $
	  \State$mejorSol(i,i+2).W_1$\Ode{1}
	  \State $W_2' \leftarrow mejorSol.W_2 - mejorSol(i,i+1).W_2 - mrjorSol(i+1,i+2).W_2 + $
	  \State $mejorSol(i,i+2).W_2$\Ode{1}
	  \If{$W_1' < k \wedge (mejorVecino.W_1 \geq k \vee W_2' < mejorVecino.W_2)$}\Ode{1}
	    \State $mejorVecino = (mejorSol_1, ... ,mejorSol_i, mejorSol_{i+1}, ..., mejorSol_{cantEjes})$\Ode{1}
	  \EndIf
      \EndIf
      \State $i = i+1$\Ode{1}
   \EndWhile
   \EndProcedure
  \end{pseudo}
 \end{figure}
\end{flushleft}

\begin{flushleft}
 \begin{figure}[H]
  \begin{pseudo}
   \Procedure{buscarVecinoCambiar}{mejorSol, mejorVecino, G}
   \State $i \leftarrow 0$\Ode{1}
   \While{$0 \leq j < cantEjes(mejorSol)-2$}\hfill$n*O(1)$
      \For{$j \in adyacentes(mejorSol_i,G) \cap adyacentes(mejorSol_{i+2},G)$}\hfill$n*O(1)$
	  \State $W_1' \leftarrow mejorSol.W_1 - mejorSol(i,i+1).W_1 - mejorSol(i+1,i+2).W_1 + (mejorSol_i, j).W_1 + $
	  \State $(j, mejorSol_{i+2}).W_1$\Ode{1}
	  \State $W_2' \leftarrow mejorSol.W_2 - mejorSol(i,i+1).W_2 - mejorSol(i+1,i+2).W_2 + (mejorSol_i, j).W_2 + $
	  \State $(j, mejorSol_{i+2}).W_2$\Ode{1}
	  \If{$W_1' < k \wedge (mejorVecino.W_1 \geq k \vee W_2' < mejorVecino.W_2)$}\Ode{1}
	    \State $mejorVecino = (mejorSol_1, ... ,mejorSol_i, j ,mejorSol_{i+2}, ..., mejorSol_{cantEjes})$\Ode{1}
	  \EndIf
      \EndFor
      \State $i = i+1$\Ode{1}
   \EndWhile
   \EndProcedure
  \end{pseudo}
 \end{figure}
\end{flushleft}
%En este algoritmo usamos la estructura Camino, que es una lista de ejes y además tiene el peso total $W_1$ y $W_2$. Y además usamos la estructura vecino, que tiene tres tipos: $TIPO1$ significa que es un vecino que surge de eliminar un nodo de la solución actual, $TIPO2$ que surge de agregar un nodo, o $TIPO3$, que surge de cambiar un nodo por otro. Además la estructura tiene una posición donde hay que hacer los cambios para pasar del camino actual al vecino (estos cambios se deducen según el tipo del vecino que sea). Por último la estructura cuenta con una lista de nodos a agregar al camino actual, si se cambia o agrega un nodo la lista tendrá un elemento y si se elimina un nodo la lista será vacía.

%Utilizamos esta estructura para poder pasar de un camino a su vecino en una complejidad constante ya que ésta cuenta con la información necesaria para saber cuáles son los cambios a realizar y dónde hay que realizarlos. Si no contáramos con esta estructura y tuviésemos simplemente otro camino deberíamos copiar el camino para actualizar la solución, lo que tendría una complejidad temporal lineal.