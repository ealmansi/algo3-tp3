La esencia de este algoritmo se encuentra principalmente en la vecindad que elegimos para las posibles soluciones, la cual detallaremos a continuación.

Supongamos que tenemos un camino $C$ = ($c_0$, $c_1$, ..., $c_k$) de $k+1$ nodos que es una solución factible del problema de CACM par un grafo $G$ = ($V$,$E$). Los vecinos de la solución $C$ son caminos que también sean soluciones factibles y que surjan de aplicarle una de las siguientes operaciones:

\begin{itemize}
 \item Quitar Nodo:

  Supongamos que existe un $i$ tal que $0 < i < k$ y que ($c_{i+1}$, $c_{i+1}$) $\in$ $E$. Bajo estas condiciones podemos afirmar que existe un camino en $G$ llamado $C'$ tal que $C'$ = ($c_0$, ..., $c_{i-1}$, $c_{i+1}$, ..., $c_k$). Esto significa que dado $C$ podemos eliminar el nodo $c_i$ del camino y unir $c_{i-1}$ y $c_{i+1}$ ya que estos son adyacentes entre sí, obteniendo un camino $C'$ de $k$ nodos.
  \item Agregar Nodo:

  La operación agregar nodo es la operación inversa a la operación Quitar Nodo descripta anteriormente. Supongamos que existe un $i$ tal que $0 \leq i < k$ y existe un nodo $x \in V$ tal que ($c_i$,$x$) $\in E$ y ($x$, $c_{i+1}$) $\in E$. En este caso podemos afirmar que podemos construir un camino $C'$ = ($c_0$, ..., $c_{i}$,$x$,$c_{i+1}$) de $k+2$ nodos.

  Vale aclarar que para realizar esta operación, el nodo $x$ no debe pertenecer a $C$, ya que queremos que el camino sea simple.

  \item Cambiar Nodo:

  Esta operación es una mezcla de las dos operaciones anteriores. Supongamos que existe un $i$ tal que $0 \leq i < k-1$ y que existe un nodo $x \in V$ $\wedge$ $x$ $\notin$ $C$ tal que ($c_i$,$x$) $\in E$ y ($x$, $c_{i+2}$) $\in E$. En este caso podemos cambiar el nodo $c_{i+1}$ del camino $C$ por el nodo $x$ obteniendo un nuevo camino $C'$ de la misma cantidad de nodos.

  Uno podría creer que la operación Cambiar Nodo es una combinación de las dos operaciones descriptas anteriormente Agregar Nodo y Quitar Nodo, ya que se podría quitar primero el nodo $c_{i+1}$ y luego agregar el nodo $x$. Sin embargo, esto no es así ya que podría pasar que el eje ($c_i$,$c_{i+2}$) $\notin E$ y, si esto ocurre no nos sería posible quitar el nodo $c_{i+1}$ utilizando Quitar Nodo para luego agregar el nodo $x$ utilizando Agregar Nodo.

\end{itemize}
 
Dadas estas tres operaciones, un vecino de una solución factible $C$ es un camino que se pueda construir aplicándole a $C$ una de estas tres operaciones. Recordar que si existiese un $C'$ que se pudiera construir a partir de aplicarle a $C$ más de una operación, $C'$ no sería vecino de $C$.

Tomamos esta decisión debido a que pensamos que si consideráramos vecino a un camino que se pudiera construir a partir de aplicarle numerosas operaciones a $C$ estaríamos perdiendo, por así decirlo, la localidad de la búsqueda y estaríamos incurriendo en una búsqueda más global. También nos parece importante recalcar que sólo tomamos como un vecino de $C$ un camino $C'$ factible.

Algo que pensamos mientras elegíamos nuestra vecindad es que una buena propiedad que ésta pudiese cumplir sería que, dada una solución factible $C$, aplicándole una cantidad de operaciones, pudiésemos llegar a cualquier otra solución factible.

Para explicar esto pensemos en los números reales. Supongamos tenemos una función $F: \mathbb{R} \Rightarrow \mathbb{R}$ y queremos hallar su máximo mediante una búsqueda local. Para esto podemos elegir, dado un $c \in \mathbb{R}$, una función de vecindad que devuelva por ejemplo el conjunto $[c-\varepsilon, c+\varepsilon]$  para algún $\varepsilon$. Dada esta función de vecindad podemos observar que aplicándola muchas veces nos podemos mover a lo largo de todo el dominio de la función $F$.

Que una función de vecindad cumpla con esta propiedad nos parece bueno porque al poder moverse de cualquier solución factible a cualquier otra, hay menos probabilidad de perder soluciones factibles buenas. Más aún, si esta propiedad vale y tuviésemos un oráculo que nos dijera que vecino tomar, podemos afirmar que si aplicamos muchas veces la función de vecindad (empezando de una solución $C$ luego en un vecino de $C$ y así) eventualmente llegaremos a la solución óptima del problema.

Luego de pensarlo, llegamos a la conclusión de que nuestra vecindad no cumple con esa propiedad, ya que existen casos en los que a partir de una solución nos es imposible llegar a otra.

Supongamos que nuestro grafo es $C_n$, con $n$ $>$ $3$ y que nuestro nodo inicial es $c_1$ y nuestro nodo destino es $c_n$. Supongamos además que tenemos una solución que es $S$ = ($c_1$, $c_2$, ... , $c_n$). Podemos observar que hay otra posible solución, que es $S'$ = ($c_1$,$c_n$) ya que como nuestro grafo es $C_n$ entonces $c_1$ y $c_n$ son adyacentes. Sin embargo, con nuestra función de vecindad nos es imposible llegar de la solución $S$ a la solución $S'$. ya que no hay ninguna operación que se pueda realizar a $S$.

Supongamos ahora que tenemos como solución inicial $S'$. En este caso tampoco podemos aplicar ninguna operación a $S'$ y no podremos llegar a la otra solución. Podemos decir que en este caso, en nuestra vecindad tanto $S$ como $S'$ son soluciones sin vecinos.

Luego de pensar como solucionar este problema, llegamos a la conclusión de que para resolverlo podríamos realizar las mismas operaciones pero en vez de a un nodo, de a $k$ con un $k$ fijo. Esto genera una vecindad muy grande y se podría perder la localidad de la búsqueda. No solo eso, sino también para casos de caminos muy disjuntos seguiría pasando lo mismo. Por ejemplo, en el caso anterior, cuando $n >> k$ seguiría pasando lo mismo.
